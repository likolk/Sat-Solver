\documentclass[12pt,a4paper]{article}
\usepackage{amssymb, amsmath}
\usepackage{theorem}
\usepackage{clrscode}
\usepackage{fullpage}
\usepackage{amsmath}

\usepackage[document]{ragged2e}

\setlength{\parskip}{0cm}
\setlength{\parindent}{0pt}
\newtheorem{thm}{Theorem}

\pagestyle{empty}

\begin{document}

\begin{center}

  \bigskip \bigskip

  {\large \textbf{ToC Spring 2022}}

   \bigskip

  % {\large {Due: Monday, 30 March 2020, 10:30 a.m.}}

  %  \bigskip
   {\large \textbf{HOMEWORK 6 REPORT}}



  %  \bigskip

  %  {\large Homework 6 Report}



  \bigskip \bigskip

  % {\large \textsc{Alexandra Willi}}

  {\large Bettelini Carlo, Likollari Kelvin, Milanesi Claudio, \\
  Alexandra Willi
  }

   \bigskip



  \bigskip \bigskip



  \bigskip \bigskip

\end{center}

\section*{Introduction}

As the name suggests, SAT solvers can be used to find a solution to a problem. Because the solver has its own language, the problem and the set of constraints needs to be encoded into the solver's language before using the solver.
The idea is to encode each constraint as a set of clauses (consisting of $ORs$ of variables) that will then be concatenated, creating a single CNF for which the solver will try to find a solution.
Upon feeding the CNF to the solver, if the CNF is satisfiable, the solver will return the values of the variables for which the CNF evaluates to true.
For the Sudoku satisfiability problem, the following conditions must hold:
\begin{enumerate}
  \item Each cell must have at least a number  between 1 and 9
  \item Each number must appear only once in each row
  \item Each number must appear only once in each column
  \item Each number must appear only once in each 3 by 3 box
  \item The initial input numbers must be present
\end{enumerate}

\section*{Implementation}
To encode the problem, we chose to denote each variable as $InCell(i,j,n)$, where $i$ denotes the row, $j$ denotes the column and $n$ denotes the number in the cell.
Since each cell can have a value from 1 to 9, and all cells need to be filled, we have up to $9\cdot9\cdot9 = 729$ different variables.
Furthermore, since the SAT solver we chose, Z3, works with integers as variables in the clauses, we created a map from each variable to its unique ID.

Upon creating the mapping, we encoded the constraints into clauses as follows: the first condition can be encoded straight forward. What this means is that for each cell $(i,j)$, a number between 1 and 9 must be present. We can write the clauses as follows:
\begin{align*}
  InCell(1,1,1) \vee InCell(1,1,2) \vee InCell(1,1,3) \vee ...
\end{align*}

For the second condition, saying that each number must appear only once in each row is equivalent to saying that a same number cannot stay in the same row. For the first row, the clause can be written as follows:
\begin{align*}
  \neg(InCell(1,1,1) \wedge InCell(1,2,1) \wedge InCell(1,3,1) \wedge ...)
\end{align*}
which is equivalent to:
\begin{align*}
  \neg InCell(1,1,1) \vee \neg InCell(1,2,1) \vee \neg InCell(1,3,1) \vee ...
\end{align*}
The same thing is then be repeated for all other rows.
\medskip

This reasoning can be used for the third condition as well, but instead of incrementing $j$, we increment $i$ first, obtaining a clause like the following for the first column:
\begin{align*}
  \neg InCell(1,1,1) \vee \neg InCell(2,1,1) \vee \neg InCell(2,1,1) \vee ...
\end{align*}
Like for the rows, we repeat this process for all columns.
\medskip

For the fourth constraint, we need to check that each number in each box is unique. Similarly as above, this is like saying that each number cannot be in the same box more than once. For the first box and the number one for example, we can write the clause as follows:
\begin{align*}
  \neg InCell(1,1,1) \vee \neg InCell(1,2,1) \vee \neg InCell(1,3,1) \vee \\
  \neg InCell(2,1,1) \vee \neg InCell(2,2,1) \vee \neg InCell(2,3,1) \vee \\
  \neg InCell(3,1,1) \vee \neg InCell(3,2,1) \vee \neg InCell(3,3,1) \\
\end{align*}
Once again we repeat this process for each remaining box.
\medskip

For the last condition, we just need to write one clause for each variable we have in the input. For example, given number 3 in cell $(1,2)$ and number 6 in cell $(5,5)$, we write:
\begin{align*}
  InCell(1,2,3) \wedge InCell(5,5,6)
\end{align*}
where each variable denotes a clause.

\medskip

Upon generating all clauses, we feed them to the Z3 solver which will $AND$ them and return a solution if the obtained CNF is solvable.
A thing to note is that since the solver works with indexes and not with strings, instead of using the string representation of a variable in the clause, we use it's associated ID using the mapping we implemented previously.
Also, for it to work we need to pass the clauses in a specific manner, like shown in the slides seen in class.

\end{document}
